% Created 2022-12-01 Thu 09:40
% Intended LaTeX compiler: pdflatex
\documentclass[11pt]{article}
\usepackage[utf8]{inputenc}
\usepackage[T1]{fontenc}
\usepackage{graphicx}
\usepackage{grffile}
\usepackage{longtable}
\usepackage{wrapfig}
\usepackage{rotating}
\usepackage[normalem]{ulem}
\usepackage{amsmath}
\usepackage{textcomp}
\usepackage{amssymb}
\usepackage{capt-of}
\usepackage{hyperref}
\author{Docker image runner}
\date{\today}
\title{Advent of Code}
\hypersetup{
 pdfauthor={Docker image runner},
 pdftitle={Advent of Code},
 pdfkeywords={},
 pdfsubject={My inconsistent advent of code solutions in varioues languages},
 pdfcreator={Emacs 27.1 (Org mode 9.3)}, 
 pdflang={English}}
\begin{document}

\maketitle
\tableofcontents

\section{Day 1}
\label{sec:org22ba0e3}
\subsection{Emacs Lisp}
\label{sec:org4416c21}
\begin{verbatim}
(defun sorted-calories (input-file)
  "Read the file and return a sorted list of elf item calories."
  (with-temp-buffer
    (insert-file-contents input-file)
    (let* ((file-content (buffer-string))
	   (list (split-string file-content "\n\n"))
	   (list-of-lists (mapcar
			   (lambda (list) (split-string list "\n"))
			   list))
	   (parsed (mapcar (apply-partially #'mapcar #'string-to-number) list-of-lists))
	   (summed-up (mapcar (apply-partially #'cl-reduce #'+) parsed)))
      (sort summed-up #'>))))

(let* ((sorted (sorted-calories "/tmp/input.txt"))
       (part-1-solution (car sorted))
       (part-2-solution
	(apply #'+ (seq-take sorted 3))))
  (print (list part-1-solution part-2-solution)))
\end{verbatim}

\subsection{Haskell}
\label{sec:orgb44daa0}
\begin{verbatim}
import Data.List (sort)
import Data.List.Split (splitOn)

sortedCalories :: String -> [Int]
sortedCalories =
    (reverse . sort . (map sum) . (map (map read)) . (splitOn [""]) . lines)

highest = head . sortedCalories

top3highest = sum . (take 3) . sortedCalories

main = do
  file <- readFile "/tmp/input.txt"
  -- Part 1 solution
  print . highest $ file
  -- Part 2 solution
  print . top3highest $ file
\end{verbatim}
\end{document}
